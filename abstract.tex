\begin{abstract}
System logs are essential for operational monitoring, security analysis, and compliance auditing, yet their unstructured nature demands substantial storage infrastructure. While generic compressors like gzip achieve 10--15$\times$ compression through byte-stream pattern matching, they treat logs as undifferentiated text, missing opportunities to exploit semantic structure inherent in log formats. This thesis presents LogPress, a semantic-aware compression system that automatically discovers log schemas through constraint-based pattern matching and applies field-specific encoding strategies tailored to data types---delta encoding for timestamps, dictionary compression for categorical fields, and varint encoding for numeric values. Unlike machine learning approaches requiring training data, LogPress extracts schemas purely through regex-based semantic classification and log alignment algorithms, enabling deployment without manual annotation overhead. Evaluation across 8 real-world datasets totaling 1,072,831 logs (131 MB uncompressed) from diverse domains---web servers, mobile applications, distributed systems, and cloud infrastructure---demonstrates that LogPress achieves 12.2$\times$ average compression ratio, reaching 98.4\% of gzip-9's efficiency while enabling 6.5$\times$ faster query execution through selective field-level decompression. On structured logs with low-cardinality categorical fields, LogPress outperforms gzip by up to 1.66$\times$ (OpenStack: 20.8$\times$ vs 12.5$\times$), though it underperforms on message-heavy logs where substring repetition favors generic algorithms. LogPress demonstrates that semantic compression provides practical advantages for log archival scenarios where storage reduction and queryability justify moderate throughput costs, advancing log management capabilities for high-volume production environments where compressed archives are queried repeatedly without full decompression.
\end{abstract}
